\documentclass[12pt,a4paper,oneside]{report}
\usepackage[utf8]{inputenc}
\usepackage{amsmath}
\usepackage{amsfonts}
\usepackage{amssymb}
\usepackage{graphicx}
\usepackage[left=3.00cm, right=3.00cm, top=3.00cm, bottom=3.00cm]{geometry}
\usepackage[brazilian]{babel}
\author{Paulo Batista da Costa \and Luan Bodner do Rosário}
\title{BodyFitness -- Documentação}
\begin{document}
	\maketitle
	\section*{1 -- Requisitos Funcionais}

	\begin{enumerate}
		
		 \item O sistema permitirá acesso de administrador
		 \item O sistema permitirá acesso de gerente
		 \item O sistema permitirá cadastro de clientes
		 \item O sistema permitirá cadastro de funcionários
		 \item O sistema permitirá gerenciamento de pagamento dos clientes
		 \item O sistema autorizará a entrada de clientes na academia
		 \item O sistema gerará um relatório mensal contendo clientes adimplentes e inadimplentes 
		 \item O sistema permitirá elaboração de balanço mensal
		 \item O sistema gerará relatório de balanço mensal
		 \item O sistema gerará uma senha exclusiva por cliente
		 \item O sistema permitirá cadastro de treinos físicos
		 \item O sitema permitirá cadastro de avaliações físicas dos clientes
		 \item O sistema associará treinos físicos a avaliações físicas para cada cliente
		 \item O sistema gerará relatório contendo resultado da avaliação física do cliente
		 \item O sistema marcará datas para as avaliações físicas do cliente
		 \item O sistema marcará datas para períodos de treino
		 \item O sistema guardará dados de funcionários
		 \item O sistema permitirá cadastro de equipamentos da academia
		 \item O sistema marcará datas de manutenção para cada equipamento
		 \item O sistema permitirá cadastro de aulas de ginástica
		 \item O sistema permitirá agendamento de aulas de ginástica
	

	\end{enumerate}
	\section*{3 -- Descrição de Casos de Uso}
	\subsection*{3.1 -- Logar No sistema}
	\textbf{Entrada:} Usuário, senha 
	
	\textbf{Descrição:} Para utilizar o sistema, o ator -- no caso o administrador ou pessoa que ocupe função de gerente -- acessará o sistema através de um nome de usuário e uma senha que lhe conferirá permissões para operar o sistema.
	
	\subsection*{3.2 -- Cadastrar Cliente}
	\subsubsection*{3.2.1 -- Inserir Cliente ao Cadastro}
	\textbf{Entrada:} Dados de usuário

	\textbf{Descrição:} Este caso de uso se refere ao cadastro de um cliente. Para cadastrar alguém como cliente do estabelecimento são necessários alguns dados, dentre os quais estão: 
	
	\begin{itemize}
		\item Nome completo
		\item Data de nascimento
		\item Endereço completo (rua, bairro, cep, número, cidade)
		\item Telefone para contato
		\item Profissão
		\item Número do RG
		\item Número do CPF
	\end{itemize}
	Estes dados são utilizados na ficha cadastral do cliente, eles são para estabelecer fácil acesso aos clientes por parte do administrador/gerente.
	\subsubsection*{3.2.2 -- Gerar Senha Exclusiva para Cliente}
	\textbf{Descrição:} No momento do cadastro, o sistema fornecerá uma senha exclusiva ao cliente a qual será utilizada para acessar as dependências do estabelecimento.
	\subsection*{3.3 -- Cadastrar Funcionário}
	\textbf{Entrada: Dados de funcionário,Cargo}
	
	\textbf{Descrição:} Este caso de uso é referente ao ato do administrador/gerente cadastrar um funcionário no sistema. Os dados para tal operação são: 
	
	\begin{itemize}
		\item Nome completo
		\item Data de nascimento
		\item Endereço completo 
		\item Número do CPF
		\item Número do RG
		\item Cargo ocupado na academia
		\item Salário
	\end{itemize} 
	\textbf{Observação:} Aqui é necessário salientar que caso o funcionário em questão seja ocupante do cargo de gerente, para o mesmo será elaborada uma conta de gerente para este operar sobre o sistema.
	
	\subsection*{3.4 -- Cadastrar Treino}
	\textbf{Entrada:} Nome do treino, grupo de músculos treinados
	
	\textbf{Descrição:} Este caso de uso é referente a cadastro de treinamentos. Para cadastrar um treino são necessários alguns dados, dentre os quais estão:
	
	\begin{itemize}
		\item Nome do treino
		\item Grupo de músculo a ser treinado
		\item Equipamentos utilizados
		\item Séries por exercício/equipamentos
	\end{itemize}
	\subsection*{3.5 -- Cadastro de Equipamento}
	\subsubsection*{3.5.1 -- Inserir equipamento}
	\textbf{Entrada:} Nome do equipamento, data de aquisição, intervalo de manutenção
	
	\textbf{Descrição:} Este caso de uso é referente a realização de cadastro de um equipamento presente no estabelecimento. Para o cadastro, são necessários apenas dois dados: Nome do equipamento e data de aquisição do mesmo. 
	\subsubsection*{3.5.2 -- Gerar Data de Manutenção}
	\textbf{Descrição:} A partir do cadastro e da inserção do intervalo de manutenção, o sistema deverá avisar aos administradores a necessidade de enviar um equipamento para a manutenção.
	
	
	\textbf{Observação:} Em eventualidade do fato de um equipamento não se encontrar em situação de uso, este poderá ser excluído do cadastro de equipamentos da academia.

	\subsection*{3.6 -- Atualizar Condicionamento de Cliente}
	\textbf{Entrada:} Dados de Condicionamento Físico (Referenciado em Apêndice A)
	
	\textbf{Descrição:} Este caso de uso se refere à inserção dos dados referentes ao condicionamento físico do cliente. 
	\subsection*{3.7 -- Associar Condicionamento Físico à Treino}
	\textbf{Descrição:} Este caso de uso é decorrente da atualização do estado de condicionamento físico do cliente. Após a atualização, o sistema deverá associar o condicionamento físico corrente com um treino previamente cadastrado.
	
	\subsection*{3.8 -- Definir Período de Treino}
	\textbf{Descrição:} Este caso de uso é decorrente da associação do condicionamento físico a um treino. Aqui é definido por quanto tempo o cliente ficará submetido a um determinado treino.
	
	\subsection*{3.9 -- Definir Nova Data de Avaliação}
	\textbf{Descrição:} Neste caso de uso, o sistema deverá gerar uma nova data de avaliação física para o cliente. Isto com base na sua condição física e ao treino ao qual ele está submetido.
	
	\subsection*{4.0 -- Gerar Relatório de Condicionamento Físico}
	\textbf{Descrição:} Neste caso de uso, deve ser dado ao administrador/gerente a possibilidade de gerar relatórios das condições de condicionamento físico de clientes a pedido deste. Este deverá conter os dados cadastrais do cliente, bem como, os dados correspondentes ao condicionamento físico do cliente. O formato de saída do documento será o \textbf{PDF}.
	
	\subsection*{4.1 -- Cadastrar Aula}
	\textbf{Entrada:} Nome da aula
	
	\textbf{Descrição:} Neste caso de uso, o usuário (administrador/gerente) terá a possibilidade de cadastrar um tipo de aula.
	
	\subsection*{4.2 -- Agendar aulas}
	\textbf{Entrada:} Aula, Horário/Data, Funcionário responsável
	
	\textbf{Descrição:} Neste caso de uso, o administrador/gerente poderá agendar uma aula tendo como entrada a aula que é cadastrada a partir do caso de uso ``4.1", o horário e a data, e o funcionário responsável. Aqui, também  será inserido os clientes que participarão da mesma.
	
	\subsection*{4.3 -- Registrar Pagamento de Cliente}
	\textbf{Entrada:} Cliente, Valor do Pagamento, Responsável pelo Recebimento
	
	\textbf{Descrição:} Neste caso de uso, o cliente realizará o pagamento tendo a dívida mensal quitada. 
	
	\subsection*{4.4 -- Autorizar Entrada de Cliente no Estabelecimento}
	\textbf{Entrada:} Senha de entrada
	\textbf{Descrição:} Neste caso de uso, o cliente acessa a entrada com sua senha. Caso o cliente seja inadimplente, sua entrada é recusada. Caso contrário, terá entrada permitida.
	
	\subsection*{4.5 -- Gerar Relatório de Clientes Inadimplentes e Adimplentes} 
	\textbf{Descrição:} Neste caso de uso, o sistema gerará um relatório de clientes que estão adimplentes e inadimplentes. Este relatório é solicitado pelo gerente/administrador. Seu formato será o \textbf{PDF}.
	
	
\section*{Tabelas}
\label{tabelas}
\begin{table}[!h]
\centering
\begin{tabular}{|c|c|}

\hline Peso Atual (Kg) & Altura (m) \\ 
\hline IMC & Nível IMC \\ 
\hline Gorgura ideal ( \%) & Gordura Atual (\%) \\ 
\hline Massa Magra (Kg) & Massa Gorda (Kg) \\ 
\hline 

\end{tabular}
\caption{Composição Corporal - Índices}
\vspace{1cm}

\begin{tabular}{|c|c|c|}
\hline Subescapular (mm) & Axilar-média (mm) & Abdominal (mm) \\ 
\hline Bicipital (mm) & Supra-ilíaca (mm) & Coxa (mm) \\ 
\hline Tricipital (mm) & Peitoral (mm) & Panturrilha (mm) \\ 
\hline 

\end{tabular} 
\caption{Composição Corporal - Dobras Cutâneas - Perímetros}
\vspace{1cm}

\begin{tabular}{|c|c|}
\hline Ombro (cm) & Pescoço (cm) \\ 
\hline Tórax Relaxado (cm) & Tórax Inspirado (cm) \\ 
\hline Abdome (cm) & Cintura (cm) \\ 
\hline Quadril (cm) & Relação - Cintura X Quadril \\ 
\hline 

\end{tabular} 
\caption{Composição Corporal - Membros Superiores | Perímetros}
\vspace{1cm}
\begin{tabular}{|c|c|c|c|}
\hline Direito(a) & Antebraço (cm) & Braço Contraído (cm) & Panturrilha (cm) \\ 
\hline Esquerda(a) & Braço Relaxado (cm) & Coxa (cm) &  \\ 
\hline 

\end{tabular} 
\caption{Composição Corporal - Membros Inferiores | Perímetros}
\end{table}
		
\end{document}